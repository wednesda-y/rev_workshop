\documentclass{beamer}
\usetheme{Madrid}
\usepackage{tikz-cd}
\usepackage{listings}
\lstset{basicstyle=\footnotesize\ttfamily,breaklines=true}
\begin{document}
\title{Introduction to x86 Reverse Engineering}
\author{Melbourne Information Security Club}
\date{March 16, 2020}
\frame{\titlepage}
\begin{frame}[fragile]
  \frametitle{What is Reverse Engineering?}
  If we consider the ordinary process of \textit{engineering} in the
  context of software.
  \[
  \begin{tikzcd}
    \text{Conceptual Idea or Design} \arrow[d, "\text{Development}"]              \\
    {\text{High Level Code, (Python, Java, C etc.)}} \arrow[d, "\text{Compiler}"] \\
    \text{Machine Code / Assembly}
  \end{tikzcd}
  \]
\end{frame}

\begin{frame}[fragile]
  \frametitle{What is Reverse Engineering?}
  Reverse Engineering is the opposite process.
  \[
  \begin{tikzcd}
    \text{Conceptual Idea or Design}                           \\
    \text{Machine Code/ Assembly} \arrow[u]
  \end{tikzcd}
  \]
  \begin{itemize}
    \item That is, the end goal is to work backwards from a
    non human-friendly form of a program, in order to gain a conceptual understanding of what the
    program is doing.
    \item This workshop will teach you how to read (32-bit) x86 assembly in order
    to reverse engineer what a given program is doing.
  \end{itemize}
\end{frame}

\begin{frame}[fragile]
  \frametitle{How to disassemble Linux ELF Binaries}
  We first need to understand how to execute ELF binaries.
  \begin{example}
    Given an ELF Binary,
    \begin{lstlisting}[basicstyle=\small\ttfamily]
    $ file behemoth0
    behemoth0: setuid ELF 32-bit LSB executable, Intel 80386, version 1 (SYSV),
    \end{lstlisting}
    We first make sure it is executable
    \begin{lstlisting}[basicstyle=\small\ttfamily]
    $ chmod +x behemoth0
    $ ls -l behemoth0
    -r-sr-x--x 1 darrenx darrenx 5900 Aug  8  2019 behemoth0
    \end{lstlisting}
    and then we can run it using \texttt{./}
    \begin{lstlisting}[basicstyle=\small\ttfamily]
    $ ./behemoth0

    \end{lstlisting}
  \end{example}
\end{frame}
\begin{frame}
  \frametitle{How to disassemble Linux ELF Binaries}
  Now that we know how to execute binaries, we will now use a debugger to
  look at the underlying assembly that we will need to reverse engineer.
  \begin{block}{Using \texttt{gdb} to disassemble a binary}
    Basic \texttt{gdb} commands summary sheet:
    \begin{itemize}
      \item View functions: \texttt{(gdb) info functions}
      \item Change disassembly from AT\&T syntax to Intel:
      \texttt{(gdb) set disasembly-flavor intel}
      \item Disassemble function (e.g main): \texttt{(gdb) disass main}
    \end{itemize}
  \end{block}
  \begin{exampleblock}{Demo}
    Use \texttt{gdb} to disassemble the main function of a \texttt{hello\_world}
    program.
  \end{exampleblock}
\end{frame}

\begin{frame}
  \frametitle{Introduction to x86 Assembly}
  There are two fundamental concepts which you'll need to know.
  \begin{definition}
    A \textbf{register} is a quickly accessible storage location, located
    directly on the processor itself. We can think of them as
    essentially \textit{untyped variables}
  \end{definition}
  \begin{itemize}
    \item The 32-bit general purpose registers we generally deal with are given the names
    EAX, EBX, ECX, EDX, ESI, EDI, ESP, EBP.
    \item The EIP register points to the next instruction to be executed.
    \item The ESP and EBP registers point to the top of the stack and the bottom
    of the current stack frame respectively.
    \item Although it is technically possible for a general purpose register to hold any 32-bit piece
    of data, registers are typically associated with a special purpose.
  \end{itemize}
  \begin{example}
    For example,EAX typically stores the result of arithmetic operations.
  \end{example}
\end{frame}

\begin{frame}
  \frametitle{Introduction to x86 Assembly}
  \begin{definition}
    \textbf{Instructions} are the basic statements that tell the processor what to do.
  \end{definition}
  \begin{examples}
    \begin{itemize}
      \item \texttt{mov eax, ecx} moves (or more accurately copies) data from
      the ECX register to the EAX register.
      \item \texttt{push eax} or \texttt{pop eax} places the value of EAX onto
      the stack, or removes the value at the top of the stack and places it into EAX
      respectively.

      \item \texttt{add eax, ebx} adds the values of the EAX and EBX registers,
      and stores the result in the EAX register. Other arithmetic operations are
      available, and in Intel syntax, the result is stored in the left operand.
    \end{itemize}
  \end{examples}
\end{frame}

\begin{frame}
  \frametitle{Recognising Code Patterns}
  So technically, we now possess the requisite knowledge to reverse engineer
  binaries. Given any disassembled function we simply need to
  \begin{itemize}
    \item Read through every single
    instruction one at a time,
    \item figure out what effect the instruction has on each register,
    \item and we can just look up any instructions we don't know in the (2000 page long) Intel x86 Software Developer's Manual.
    \item .... right??
  \end{itemize}

  \begin{alertblock}{Main takeaway from this workshop}
    Trying to reverse engineer a program by reading through and interpreting
    every single instruction is ridiculous. What is more important to gaining
    a conceptual understanding of the program is to \textbf{recognise code patterns}.
  \end{alertblock}
\end{frame}

\begin{frame}[fragile]
  \frametitle{The Simplest Example}
  The simplest program we can do is an empty function.
  \begin{block}{C code}
  \begin{lstlisting}[language=C, basicstyle=\small\ttfamily]
  void func(){
    ;
  }
  \end{lstlisting}
\end{block}
When we look at the compiled binary in a disassembler, we get
\begin{block}{Assembly}
  \begin{lstlisting}[basicstyle=\small\ttfamily, language={[x86masm]Assembler}]
(fcn) sym.func 2
  0x000004f0      f3c3           ret
\end{lstlisting}
\end{block}
When a function is called, the return address is pushed onto the stack.
At the end of a function, the \texttt{ret} instruction causes the program to jump
to the address located at top of the stack.
\end{frame}

\begin{frame}[fragile]
  \frametitle{Returning values}
  Lets try returning some values now.
  \begin{block}{C Code}
  \begin{lstlisting}[language=C, basicstyle=\small\ttfamily]
  int func(){
    return 1337;
  }
  \end{lstlisting}
\end{block}

\begin{block}{Assembly}
\begin{lstlisting}[basicstyle=\ttfamily]
(fcn) sym.func 6
  0x000004ed      b839050000     mov eax, 0x539
  0x000004f2      c3             ret
\end{lstlisting}
Note: 0x539 = 1337
\end{block}
From this, we can deduce that the general convention is to store
function return values in the EAX register.
\end{frame}

\begin{frame}[fragile]
  \frametitle{The Stack}
  \begin{itemize}
    \item The stack is a region of memory used by the program to store local variables,
    function arguments, and any other data not immediately required.
    \item By convention, the bottom of the stack begins at 0xffffff, and grows
    \alert{backwards} towards lower addresses.
    \item Recall that the ESP register points to the top of the stack, and the
    EBP register points to the bottom of the current stack frame. Hence the function's stack
    frame lies between these two pointers.
    \[
    \begin{tikzcd}
      \text{ESP} \arrow[r, no head, maps to] & \text{Function Stack Frame} & \text{EBP} \arrow[l, no head, maps to]
    \end{tikzcd}
    \]
    \item Establishing the stack frame (i.e moving ESP and EBP to their correct values)
    is achieved in the \textit{function prologue} and the \textit{function epilogue}.
  \end{itemize}
\end{frame}
\begin{frame}[fragile]
  \frametitle{Function prologue and epilogue}
  Let's try a function with a local variable.
  \begin{block}{C Code}
  \begin{lstlisting}[language=C, basicstyle=\small]
  void func(){
    int a = 0;
  }
  \end{lstlisting}
\end{block}
\end{frame}
\begin{frame}[fragile]
\begin{block}{Assembly}
\begin{lstlisting}[basicstyle=\small\ttfamily]
(fcn) sym.func 15
  /* Establish func()'s stack frame */
  0x000004ed      55            push ebp
  0x000004ee      89e5          mov ebp, esp
  0x000004f0      83ec10        sub esp, 0x10
  /* Assign 0 to the local variable 'a' stored on the stack */
  0x000004fd      c745fc000000. mov dword [ebp-0x4], 0
  0x00000504      90            nop
  /* Clean up func()'s stack frame */
  0x00000505      c9            leave
  0x00000506      c3            ret
\end{lstlisting}
\end{block}
Note that \texttt{leave} is a short hand for \texttt{mov esp, ebp; pop ebp}.
\end{frame}

\begin{frame}[fragile]
  \frametitle{Passing function arguments}
  Let's try a function that takes in arguments, firstly from the callers' side
  \begin{block}{C Code}
  \begin{lstlisting}[language=C, basicstyle=\small\ttfamily]
  int main(){
    return func(1,2);
  }
  \end{lstlisting}
\end{block}
\end{frame}
\begin{frame}[fragile]
\begin{block}{Assembly}
\begin{lstlisting}[basicstyle=\small\ttfamily]
(fcn) main 17
  0x00000504      55             push ebp
  0x00000505      89e5           mov ebp, esp
  /* Push function arguments onto stack in reverse order */
  0x00000507      6a02           push 2
  0x00000509      6a01           push 1
  0x0000050e      e8d3ffffff     call sym.func
  /* Clean up the stack space used by the arguments from before */
  0x00000511      83c408         add esp, 8
  0x00000512      c9             leave
  0x00000513      c3             ret
\end{lstlisting}
\end{block}
\end{frame}

\begin{frame}[fragile]
  \frametitle{Passing function arguments}
  Now let's see what happens from the function's perspective.
  \begin{block}{C Code}
  \begin{lstlisting}[language=C]
  int func(int a, int b){
    return a + b;
  }
  \end{lstlisting}
\end{block}
\end{frame}

\begin{frame}[fragile]
\begin{block}{Assembly}
\begin{lstlisting}[basicstyle=\small\ttfamily]
(fcn) sym.func 23
  0x000004ed      55          push ebp
  0x000004ee      89e5        mov ebp, esp
  /* Function arguments are accessed from the stack, relative to func()'s stack frame. */
  0x000004fa      8b5508      mov edx, dword [ebp+0x8]
  0x000004fd      8b450c      mov eax, dword [ebp+0xc]
  0x00000500      01d0        add eax, edx
  0x00000502      5d          pop ebp
  0x00000503      c3          ret
\end{lstlisting}
\end{block}
\end{frame}

\begin{frame}[fragile]
  \frametitle{Control Flow: Selection}
  The last two basic code patterns deal with control flow.
  \begin{block}{C Code}
  \begin{lstlisting}[basicstyle=\small\ttfamily,language=C]
  int func(int a){
    if(a > 3)
      return 1;
    else
      return 0;
  }
  \end{lstlisting}
\end{block}
\end{frame}

\begin{frame}[fragile]
\begin{block}{Assembly}
\begin{lstlisting}[basicstyle=\small\ttfamily]
(fcn) sym.func 33
     0x000004ed     55          push ebp
     0x000004ee     89e5        mov ebp, esp
     0x000004fa     837d0803    cmp dword [ebp+0x8], 3
 ,=< 0x000004fe     7e07        jle 0x507
 |   0x00000500     b801000000  mov eax, 1
,==< 0x00000505     eb05        jmp 0x50c
||   ; CODE XREF from sym.func (0x4fe)
|`-> 0x00000507     b800000000  mov eax, 0
|    ; CODE XREF from sym.func (0x505)
`--> 0x0000050c     5d          pop ebp
     0x0000050d     c3          ret
\end{lstlisting}
\end{block}
\end{frame}

\begin{frame}
  \frametitle{Control Flow Graphs}
  \begin{itemize}
    \item While it was very nice of \texttt{radare2} to include arrows in the
  disassembly of the previous slide to indicate jumps,
  in more complex functions those arrows get cluttered easily.
  \item For this reason, most reverse engineering platforms will possess the capacity to produce CFGs.
  \end{itemize}
  \begin{exampleblock}{Demo}
    Use \texttt{radare2} to view a CFG of the function on the previous slide.
  \end{exampleblock}
\end{frame}

\begin{frame}[fragile]
    \frametitle{Control Flow: Repetition}
    The last code pattern we will explore is actually best represented in a CFG.
    \begin{block}{C Code}
      \begin{lstlisting}[language=C]
int func(int a){
    for(int i = 0; i < a; i++)
        a += i;
    return a;
}
    \end{lstlisting}
    \end{block}
    \begin{exampleblock}{Demo}
    Use \texttt{radare2} to view the CFG representation of a for loop.
    Identify the loop initialisation statement, the loop index variable, the guard condition, and the increment statement.
    \end{exampleblock}
\end{frame}

\begin{frame}
\frametitle{Where to from here?}
\begin{itemize}
  \item We have covered the fundamental code patterns that make up programs, and explored
  how they appear in assembly.

  \item Build up your knowledge of code patterns
  by writing your own code, and examining the corresponding disassembly like we have done today.

  \item What we have covered today forms the basis of \textbf{static analysis}.
  There is also \textbf{dynamic analysis} where one reverse engineers a program by
  executing it in a debugging environment and tracing the flow of data as the program is running.

  \item There are also many other architectures to explore beyond 32-bit x86.

  \item Practice by doing lots of CTF challenges! We have curated some for this workshop which you
      can play at \texttt{URL}.
  \item Practice by doing lots of CTF challenges!
\end{itemize}
\end{frame}
\end{document}
